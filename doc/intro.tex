\chapter{Introduccio}

%primera versio d'estructura del document
\section{Entorn i eines de treball}
Totes les mesures, execucions i optimitzacions d'aquesta pràctica s'han dut a terme sota una arquitectura Intel Core Duo. Això ens permet treballar tant a les aules com en els nostres respectius portàtils.\\

Pel manteniment del codi i per poder treballar amb més facilitat hem optat per GIT com a sistema de control de versions. Concretament, hem optat per GitHUB \url{https://github.com/} que ofereix repositoris propis gratuïtament.

Perla feina propiament dita d'optimitzacions, hem utilitzat les eines vistes durant el curs. En el cas de fer servir alguna altra eina, ho indiquem en el document.

%scripts, makefile, etc
\section{Optimitzacions amb el compilador}
Al treballar sota una arquitectura Intel, tenim l'opció de treballar amb dos compiladors:

\begin{itemize}
  \item icc (ICC) 12.0.2 20110112
  \item gcc (Ubuntu/Linaro 4.5.2-8ubuntu4) 4.5.2
\end{itemize}

Sense començar a mirar el codi i per experiència d'anteriors pràctiques/feines, treballant sota una arquitectura Intel, el seu compil·lador natiu te millor rendiment. En el següent apartat, adjuntem un benchmarking del codi, comparant els dos compil·ladors i emprant diferents flags. Pels flags del compil·lador d'Intel, hem seguit el manual d'optimitzacion d'Intel.

%-O3, -march=core2 ... etc
\section{Profiling del codi original}


\begin{table}[h!b!p!]
\caption{Benchmarking entre compiladors i flags}
\begin{center}
\begin{tabular}{lll}
\hline
 & ICC & GCC \\
\hline
-O0 &  &  \\
-O1 &  &  \\
-O2 &  &  \\
\hline
\end{tabular}
\end{center}
\label{table1}
\end{table}
