\documentclass[final,a4paper,11pt]{report}
\usepackage[utf8]{inputenc}
\usepackage{fullpage}
\usepackage{graphicx}
\usepackage{subfig}
\usepackage{color}
\usepackage{listings}
\usepackage{hyperref}
\usepackage{parskip} %space between paragraphs instead of no-space & indent

\usepackage{amsthm,amssymb}
\usepackage[fleqn]{amsmath}

\usepackage[margin=3cm]{geometry}

\newcommand{\HRule}{\rule{\linewidth}{0.5mm}}

\begin{document}

%!TEX root = final_project.tex

\begin{titlepage}
\begin{center}

\bigskip


% Upper part of the page

\textsc{\LARGE Programació Conscient}\\[0.2cm]
\textsc{\LARGE de l'Arquitectura}\\[1.5cm]
\textsc{\Large Final Project}\\[0.5cm]


% Title
\HRule \\[0.4cm]
{ \huge \bfseries Optimization of the mpeg2 encoder}\\[0.4cm]

\HRule \\[1.5cm]

\end{center}

% Author and supervisor
\begin{minipage}{0.4\textwidth}
\begin{flushleft} \large
\emph{Authors:}\\
Sara \textsc{Dueñas}\\
\url{sara85gon@gmail.com}\\[0.5cm]
Kim \textsc{Serradell} 
\url{kim.serradell@bsc.es}
\end{flushleft}
\end{minipage}

\vfill
\begin{center}
% Bottom of the page
{\large \today}
\end{center}


\end{titlepage}


\tableofcontents
\newpage


%primera versio d'estructura del document
\section{Introducció: preparació de l'entorn}

%scripts, makefile, etc
\section{Optimitzacions amb el compilador}

%-O3, -march=core2 ... etc
\section{Profiling del codi original}

%profiling del codi original -> on hem vist que triga mes per començar a optimitzar

\section{Optimitzacions}

\subsection{funció dist1 a motion.c}

%explicar quines optimitzacions apliquem, speed-ups, etc

\section{????}

\begin{thebibliography}{9}

\bibitem{ayguade2009}
  Eduard Ayguadé,
  \emph{MPI: a message-passing parallel model}.
  Class Slides,
  2009.

\bibitem{grama}
  Ananth Grama,Vipin Kumar,George Karypis,Anshul Gupta,
  \emph{An Introduction to Parallel Computing: Design and Analysis of Algorithms, 2/e}.

\bibitem{paraver}
  Barcelona Supercomputing Center,
  \emph{Paraver tutorials and User Guide}.

\end{thebibliography}

\end{document}

